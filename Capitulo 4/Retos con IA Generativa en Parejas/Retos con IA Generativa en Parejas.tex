\documentclass[12pt]{article}
\usepackage[utf8]{inputenc}
\usepackage[spanish]{babel}
\usepackage{enumitem}
\usepackage{titlesec}
\usepackage{hyperref}
\usepackage{lmodern}

% Formato de secciones
\titleformat{\section}{\Large\bfseries}{\thesection}{1em}{}
\titleformat{\subsection}{\large\bfseries}{\thesubsection}{1em}{}

\begin{document}

\begin{center}
    {\huge \textbf{Retos con IA Generativa en Parejas}} \\
    \vspace{0.3cm}
    \rule{\textwidth}{0.4pt}
\end{center}

\section*{Reto 1: ``Selfie en Monserrate''}

\textbf{Objetivo:} \\
Experimentar con la \textbf{generación y edición de imágenes con IA}, usando como base una \textbf{selfie de ambos integrantes del grupo} para crear un montaje en un lugar icónico de Colombia: el cerro de Monserrate en Bogotá.

\textbf{Instrucciones:}
\begin{enumerate}[label=\arabic*.]
    \item Formen grupos de \textbf{dos personas}.
    \item Tómense una \textbf{selfie juntos} (puede ser con el celular).
    \item Elijan \textbf{dos modelos de IA gratuitos diferentes} de la lista comparativa (Seleccionen que Ias podrian usar de acuerdo al reto).
    \item Suban su selfie a ambos modelos y generen un \textbf{montaje en Monserrate}:
    \begin{itemize}
        \item Incluyan la Basílica y el paisaje de Bogotá. Busquen en línea fotos en las que quieran basar su prompt.
        \item Pueden probar distintos estilos (realista, acuarela, cómic, futurista).
        \item Tengan cuidado de no generar muchos prompt y acabar con el saldo del modo gratuito, primero escriban el prompt y luego lo envían.
    \end{itemize}
    \item Comparen resultados entre los modelos:
    \begin{itemize}
        \item ¿Cuál mantuvo mejor la identidad de los dos?
        \item ¿Qué IA dio un resultado más estético o convincente?
        \item ¿Cuál fue más fácil de usar?
    \end{itemize}
    \item Preparen una \textbf{mini-presentación (2 minutos) Tambien pueden usar IA para esto (i.e Canva)} mostrando:
    \begin{itemize}
        \item Su selfie original.
        \item Los dos resultados obtenidos.
        \item Una breve conclusión: ¿qué modelo recomiendan y por qué?
    \end{itemize}
\end{enumerate}

\vspace{0.5cm}
\hrule
\vspace{0.5cm}

\section*{Reto 2: ``Detectives de aves''}

\textbf{Objetivo:} \\
Usar IA \textbf{multimodal} para identificar especies de aves a partir de fotos y practicar la validación de resultados.

\textbf{Instrucciones:}
\begin{enumerate}[label=\arabic*.]
    \item Cada pareja recibirá \textbf{3 fotos de aves}.
    \item Seleccionen \textbf{dos modelos de IA con visión} (Seleccione cual sería el indicado para esta tarea).
    \item Suban cada foto a la IA y pregunten:
    \begin{itemize}
        \item Nombre común.
        \item Nombre científico.
        \item Hábitat y distribución.
        \item Un dato curioso.
    \end{itemize}
    \item Verifiquen la respuesta con otra IA o con \textbf{búsqueda web integrada} (ej. Gemini, Qwen, Grok).
    \item Elaboren una \textbf{ficha comparativa} con:
    \begin{itemize}
        \item La foto original.
        \item Nombre correcto (común + científico).
        \item El modelo que dio la respuesta más confiable.
        \item Un dato curioso.
    \end{itemize}
    \item Presenten brevemente sus hallazgos al grupo.
\end{enumerate}

\textbf{Discusión final:}
\begin{itemize}
    \item ¿Qué tan confiables fueron las respuestas iniciales?
    \item ¿Qué modelo resultó más preciso?
    \item ¿Qué aprendieron sobre la importancia de verificar la información con múltiples fuentes?
\end{itemize}

\end{document}
